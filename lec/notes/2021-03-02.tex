\documentclass[12pt, leqno]{article} %% use to set typesize
\usepackage{fancyhdr}
\usepackage[sort&compress]{natbib}
\usepackage[letterpaper=true,colorlinks=true,linkcolor=black]{hyperref}

\usepackage{amsfonts}
\usepackage{amsmath}
\usepackage{amssymb}
\usepackage{color}
\usepackage{tikz}
\usepackage{pgfplots}
\usepackage{listings}
%\usepackage{courier}
%\usepackage[utf8]{inputenc}
%\usepackage[russian]{babel}

\lstset{
  numbers=left,
  basicstyle=\ttfamily\footnotesize,
  numberstyle=\tiny\color{gray},
  stepnumber=1,
  numbersep=10pt,
}

\newcommand{\iu}{\ensuremath{\mathrm{i}}}
\newcommand{\bbR}{\mathbb{R}}
\newcommand{\bbC}{\mathbb{C}}
\newcommand{\calV}{\mathcal{V}}
\newcommand{\calE}{\mathcal{E}}
\newcommand{\calG}{\mathcal{G}}
\newcommand{\calW}{\mathcal{W}}
\newcommand{\calP}{\mathcal{P}}
\newcommand{\macheps}{\epsilon_{\mathrm{mach}}}
\newcommand{\matlab}{\textsc{Matlab}}

\newcommand{\ddiag}{\operatorname{diag}}
\newcommand{\fl}{\operatorname{fl}}
\newcommand{\nnz}{\operatorname{nnz}}
\newcommand{\tr}{\operatorname{tr}}
\renewcommand{\vec}{\operatorname{vec}}

\newcommand{\vertiii}[1]{{\left\vert\kern-0.25ex\left\vert\kern-0.25ex\left\vert #1
    \right\vert\kern-0.25ex\right\vert\kern-0.25ex\right\vert}}
\newcommand{\ip}[2]{\langle #1, #2 \rangle}
\newcommand{\ipx}[2]{\left\langle #1, #2 \right\rangle}
\newcommand{\order}[1]{O( #1 )}

\newcommand{\kron}{\otimes}


\newcommand{\hdr}[1]{
  \pagestyle{fancy}
  \lhead{Bindel, Spring 2021}
  \rhead{Numerics for Data Science}
  \fancyfoot{}
  \begin{center}
    {\large{\bf #1}}
  \end{center}
  \lstset{language=matlab,columns=flexible}  
}


\begin{document}
\hdr{2021-03-02}

\section{Introduction}

Last week, we considered least squares and related regression models.
In these models, we want to predict some dependent variable $Y$ (also
called an outcome variable) as a function of some independent
variables $X$ (also called features, regressors, or explanatory
variables).  In the simplest case, we look for a linear prediction:
\[
  Y \approx Xw
\]
where $X$ is a row vector of features and $w$ is a column vector of
weights.  We can make the model a little more expressive with a linear
prediction in a higher-dimensional space:
\[
  Y \approx \psi(X) w
\]
where $\psi$ is a (nonlinear) {\em feature map} from the original vector of
dependent variables to a new, expanded set of variables\footnote{%
  We did not talk about feature maps yet in this class.  But many
  of you will have seen feature maps and ``the kernel trick'' in
  a machine learning class, and you are implicitly using feature maps
  in the polynomial fitting homework problem from last week.  We will
  discuss these ideas in more detail next week when we discuss
  function approximation.
}.
In either case, we choose $w$ by minimizing a sum of loss functions
over examples drawn from a population, maybe together with a
regularization term.  For the least squares loss function and
some standard regularizers, we can solve the resulting optimization
problem using factorization methods from numerical linear algebra.

In some cases, though, the distinction between explanatory and output
variables is not clear.  Sometimes we may know different subsets of
the variables for each experiment, and we want to fill in the rest.
Or we may know all the variables in our experiment, and want
to look for relationships between them rather than trying to make
predictions.  Or we might want to use the attributes associated
with different objects not for prediction, but to cluster the objects,
or to find outliers.  Remarkably, we can view these tasks as well
through the lens of matrix factorization.

\section{Matrix factorizations and latent variables}

We can use matrices to encode many types of data: images, graphs, user
preferences, and distributions of jointly distributed random variables
are but a few.  Often, as in our linear regression examples, the rows
of the matrix represent objects or experiments and the columns
represent associated attributes.  In other cases, as when encoding a
graph, both rows and columns represent objects, and the entries of the
matrix represent pairwise interactions.  Factoring these data matrices
can help us compress and denoise data, separate effects of different
factors, find similarities between objects, or fill in missing data.

We generally seek to approximately factor a data matrix
$A \in \bbR^{m \times n}$ as
\[
  A \approx LMR, \quad
  L \in \bbR^{m \times r},
  M \in \bbR^{r \times r},
  R \in \bbR^{r \times n}.
\]
In this view, we can think of $R$ (or $MR$) as a map from the original
attributes to a smaller set of {\em latent factors}.  Different
factorization methods differ both in the structural constraints on
$L$, $M$, and $R$ and on how the approximation is chosen.

It is worth being a little careful in how we think of these
factorizations!  In most of linear algebra, we are interested in a
matrix as a representation of a linear map or a quadratic form with
respect to some particular basis --- but we have a rich set of bases we
can choose.  For data analysis, though, we may want to restrict the
bases we consider for the sake of interpretability.  For example, if
we want to choose as our latent factors a subset of the original
factors, or if we want to enforce non-negativity in the factors, we
cannot consider arbitrary changes of basis.  Because of this, some of
the methods that we like best for interpretability are also the most
difficult to compute, as they involve an optimization problem that is
combinatorial rather than continuous in nature.  We will see this
issue repeatedly this week.

\section{A gallery of examples}

Before we turn to our first batch of factorization tools, let us first
set the stage with some concrete example applications.

\subsection{Document search and latent semantic analysis}

The {\em vector space model} was one of the early triumphs
in the field of information retrieval.  In this model,
documents are treated as ``bags of words,'' and each document
is represented as a vector of term frequencies, one for each
word in the vocabulary\footnote{%
  Very common words (stop words) and very rare words may be removed
  from the vocabulary before taking counts.
}.
There are several different ways that the term frequencies can be
computed.  We might use a binary indicator that says whether a term is
present or absent; raw count information or relative frequency; or
something affine or nonlinear (usually logarithmic) function of the
frequency.  We also usually scale by the
{\em inverse document frequency}, which measures how common the
term is across all documents. For example, a common choice is
\[
  \operatorname{idf}(t,D) = \log \frac{N}{|\mbox{documents in $D$ containing $t$}|}.
\]
The tf-idf matrix for term $t$ and document $d$ in corpus $D$ is then
\[
  \operatorname{tfidf}(t,d,D) =
  \operatorname{tf}(t,d) \cdot
  \operatorname{idf}(t,D).
\]

In {\em latent semantic indexing}, we approximate the tf-idf matrix
by the truncated singular value decomposition, and use the result to
compute a measure of query relevance.  For example, suppose the tf-idf
matrix was arranged so that each row represents a term, and each
column represents a document.  Let $q$ be the tf-idf vector for the
words in a query; for example, it might be a vector of all zeros
except for a one in the row indicating the word ``vehicle.''  To find
documents relative to the query, we would compute the row vector
\[
  r = q^T U_k \Sigma_k V_k^T
\]
and use $r_i$ as the relevance score.  If we used the full SVD, the
relevance score would be exactly the row of the original tf-idf matrix
associated with the term ``vehicle,'' which might not include highly
relevant documents in which the word ``vehicle'' never appears but
words like ``boat'' or ``truck'' or ``car'' do appear.  By using the
SVD, we ``blur out'' the specific words and get more semantically
meaningful results.

The same idea of latent semantic indexing (LSI) or latent semantic
analyis (LSA) can apply in other settings as well.  For example,
similar ideas appear in bibliometrics, where one wants to find highly
relevant papers.  But there are also some real difficulties with LSI.
One problem is that the nonlinear mapping from term counts to matrix
entries is not always easy to justify, and an appropriate choice may
require some experimentation.  Another problem with LSI is that it is
generally impossible to assign any real meaning to the factors.

\subsection{$k$-means clustering as a matrix factorization}

The $k$-means algorithm is a standard clustering method.  Given $m$
points in $n$-dimensional space (which we think of as the rows in an
$m \times n$ matrix $A$), the $k$-means algorithm repeatedly updates a set
of $k$ representative vectors $r_1, \ldots, r_k$ as follows:
\begin{itemize}
\item
  Assign each point in the data set to a cluster based on the nearest
  representative vector (e.g.~in Euclidean distance, though we could
  also look at angles).
\item
  Recompute each representative vector as the mean of all the points
  in the cluster.
\end{itemize}

In matrix terms, the $k$-means algorithm computes the factorization
\[
  A \approx L R
\]
where the rows of $R$ are the representative vectors $r_j$ and the
rows of $L$ indicate cluster membership; that is, $L_{ij}$ is $1$ if
point $i$ is in cluster $j$, and zero otherwise.  More particularly,
the $k$-means method is an example of an {\em alternating iteration}:
first we optimize $L$ while holding $R$ fixed, then we optimize $R$
while holding $L$ fixed.  The optimization minimizes the least squares
error, and it generally converges to a local minimum in practice.

\subsection{Eigenfaces, fisherfaces, and image analysis}

The method of {\em eigenfaces} (or more generally {\em eigenimages})
has been used for image recognition and classification since 
it was developed by Sirovich and Kirby in 1980.  The method
essentially extracts a low-dimensional feature representation of
images of (gray scale) faces.  The ``eigenfaces'' are computed by
principal component analysis on a collection of (gray scale, possibly
low resolution) face images, which are each laid out in the columns of
a large matrix.  Classification is done by mapping a new face into a
low-dimensional space of eigenface features, then looking for the
nearest neighbor in that space.  As with latent semantic indexing,
the method works in part because it captures identifying
features while ``blurring out'' irrelevant fine details.

An alternative to eigenfaces is {\em fisherfaces}.  Where eigenfaces
are may be written in terms of the eigenvalue decomposition of a
covariance matrix, fisherfaces come from Fisher's linear discriminant
analysis (LDA) approach.  Here we are interested in the largest
eigenvales and vectors for the generalized problem
\[
  \Sigma_b w = \lambda \Sigma w
\]
where $\Sigma$ is the common covariance for each class of (faces) examples
in the data set, and
\[
  \Sigma_b = \frac{1}{C} \sum_{j=1}^C (\mu_i-\mu) (\mu_i-\mu)
\]
is the between-class variability matrix.  Here the $\mu_i$ are the
class means and $\mu$ is the mean of the class means.  Unlike the
eigenfaces technique, which is agnostic to class labels on the images,
the fisherfaces approach tailors the choice of features to the
classification problem at hand.

\subsection{Collaborative filtering and the Netflix challenge}

In 2009, Netflix awarded a \$1M prize in a competition to beat the
accuracy of their in-house {\em collaborative filtering} algorithm to
predict how users would rate films.  One of the key ideas in
collaborative filtering is {\em matrix completion}.  The ratings are
given in a giant matrix in which rows correspond to users and columns
correspond to movies.  But as most users have not watched most movies,
only a relatively small number of the matrix entries are known.  The
idea of matrix completion is to use those few entries to learn a low
rank factorization that matches the data and can be used to predict
the remaining entries.  The intuition is that the low rank
factorization represents a mapping of users and movies into a
low-dimensional space that captures certain common attributes
(e.g.~how much action there is in the movie, or whether the tone is
light or serious).

Somewhat remarkably, one can prove that this type of reconstruction is
possible (and even reasonably straightforward to compute)
under {\em incoherence assumptions} that we will discuss on Friday.

\subsection{Anchor words and interpretable topic models}

In latent semantic indexing, we used the SVD to compute a
low-dimensional feature space to describe documents.  However, that
space is very difficult to interpret.  We would ideally like to
summarize documents in terms of their relation to meaningful topics;
the same idea also applies to other collections, such as lists of
movies or songs that we might want to characterize by genre.  In {\em
  topic modeling}, we explain each document in terms of a distribution
over of a few stopic, where each topic is in turn a (sparse)
distribution over words.  We generally insist that all these
distribution vectors are properly stochastic: that is, the entries
should be non-negative, and they should sum to one.  Hence, topic
modeling boils down to a {\em probabilistic matrix factorization
  problem}, a particular type of {\em non-negative matrix
  factorization} (NMF).

As we will see later in the week, non-negative matrix factorizations
are generally hard to compute.  However, the problem becomes much
easier if we assume that for each topic there is at least one word
that is mostly associated with that topic (and not with others).  This
word is called an {\em anchor word} for the topic.
We can find anchor words by applying the pivoted QR algorithm, which we will
turn to shortly, to a matrix of word-word co-occurrence statistics.
Once we have the anchor words, we can compute
word-topic distributions by solving {\em non-negative least squares} problems.

\section{Some basic factorization tools}

Next time, we will start our discussion of factorizations used in data analysis
with the symmetric eigenvalue problem.  This is a useful building
block on its own, particularly for low rank approximation of symmetric
matrices.  It is also useful as a prelude to another discussion of
the singular value decomposition (SVD), that Swiss Army knife of
matrix factorizations.  But both the symmetric eigenvalue decomposition
and the singular value decomposition involve a very flexible choice
of bases; as we have mentioned, this is not always ideal when we want
an interpretable model.  For interpretability, it is helpful to talk
again about pivoted QR and the closely-related interpolative
decomposition (ID), in which we choose a subset of the data matrix
columns as a basis for the column space.  We will also mention
the closely-related CUR decomposition, in which both the left and right
factors in our approximation are drawn from the columns and rows of
the data matrix.

\end{document}
