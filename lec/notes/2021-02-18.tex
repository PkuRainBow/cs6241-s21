\documentclass[12pt, leqno]{article} %% use to set typesize
\usepackage{fancyhdr}
\usepackage[sort&compress]{natbib}
\usepackage[letterpaper=true,colorlinks=true,linkcolor=black]{hyperref}

\usepackage{amsfonts}
\usepackage{amsmath}
\usepackage{amssymb}
\usepackage{color}
\usepackage{tikz}
\usepackage{pgfplots}
\usepackage{listings}
%\usepackage{courier}
%\usepackage[utf8]{inputenc}
%\usepackage[russian]{babel}

\lstset{
  numbers=left,
  basicstyle=\ttfamily\footnotesize,
  numberstyle=\tiny\color{gray},
  stepnumber=1,
  numbersep=10pt,
}

\newcommand{\iu}{\ensuremath{\mathrm{i}}}
\newcommand{\bbR}{\mathbb{R}}
\newcommand{\bbC}{\mathbb{C}}
\newcommand{\calV}{\mathcal{V}}
\newcommand{\calE}{\mathcal{E}}
\newcommand{\calG}{\mathcal{G}}
\newcommand{\calW}{\mathcal{W}}
\newcommand{\calP}{\mathcal{P}}
\newcommand{\macheps}{\epsilon_{\mathrm{mach}}}
\newcommand{\matlab}{\textsc{Matlab}}

\newcommand{\ddiag}{\operatorname{diag}}
\newcommand{\fl}{\operatorname{fl}}
\newcommand{\nnz}{\operatorname{nnz}}
\newcommand{\tr}{\operatorname{tr}}
\renewcommand{\vec}{\operatorname{vec}}

\newcommand{\vertiii}[1]{{\left\vert\kern-0.25ex\left\vert\kern-0.25ex\left\vert #1
    \right\vert\kern-0.25ex\right\vert\kern-0.25ex\right\vert}}
\newcommand{\ip}[2]{\langle #1, #2 \rangle}
\newcommand{\ipx}[2]{\left\langle #1, #2 \right\rangle}
\newcommand{\order}[1]{O( #1 )}

\newcommand{\kron}{\otimes}


\newcommand{\hdr}[1]{
  \pagestyle{fancy}
  \lhead{Bindel, Spring 2021}
  \rhead{Numerics for Data Science}
  \fancyfoot{}
  \begin{center}
    {\large{\bf #1}}
  \end{center}
  \lstset{language=matlab,columns=flexible}  
}


\begin{document}
\hdr{2021-02-18}

\section{Direct methods for large least squares}

Our brief discussion of methods for least squares has so far focused
on {\em dense} factorization methods (normal equations with Cholesky,
QR, and SVD).  For $A \in \mathbb{R}^{m \times n}$, these methods all
require $O(mn^2)$ time to set up the factorization and $O(mn)$ time to
solve the system for a particular right hand sice.  They also all
require $O(mn)$ space.  What happens when either $m$ or $n$ is so
large as to make this awkward?  We consider a few scenarios:
\begin{itemize}
\item If $m$ is large but $n$ is not too large (say on the order of a
  few hundreds, or even 1000-2000), we might still use a standard
  factorization, but arranged to be efficient on parallel machines.
  The {\em tall skinny QR} (TSQR) approach involves breaking the
  observations into groups and doing a factorization on each group,
  then recursively combining the factorizations.
\item In many cases with large $n$, the matrix $A$ is often {\em
  sparse}: that is, most of the entries of $A$ might be zero.
  Sometimes, we have that $A$ is {\em data-sparse}: that is,
  it has special structure that can be described with far fewer than
  $mn$ parameters.
\end{itemize}

In the sparse case, we are sometimes able to use {\em sparse direct}
methods.  That is, if $A$ is sparse (most of the elements are zero),
we might be able to write an economy QR factorization $A = QR$ where
$R$ is also sparse (the matrix $Q$ is usually dense and therefore not
saved, though we might be able to store in compressed form as a
product of simpler othogonal transformations).  Sparse QR isn't always
practical, as the $R$ factor is sometimes significantly denser than
$A$.  The order of the columns in $A$ can make a huge difference in
the sparsity of $R$, and so we typically would seek a factorization
$A\Pi = QR$ where $\Pi$ is a permutation matrix that reorders the
columns.

Frequently, though, sparse direct methods are simply impractical.
In this case, we turn to {\em iterative} methods.

\section{Iterative methods}

We started with a discussion of {\em direct} methods
for solving least squares problems based on matrix factorizations.
These methods have a well-understood running time, and they produce
a solution that is accurate except for roundoff effects.  For larger
or more complicated problems, though, we turn to
{\em iterative} methods that produce a series of approximation
solutions.

We will turn now to iterative methods: gradient and stochastic
gradient approaches, Newton and Gauss-Newton, and (block) coordinate
descent.  We will see additional solver ideas as we move through the
class, but these are nicely prototypical examples that illustrate
two running themes in the design of numerical methods for
optimization.

\paragraph{Fixed point iterations}
All our nonlinear solvers (and some of our linear solvers) will be
{\em iterative}.  We can write most as {\em fixed point iterations}
\begin{equation}
  x^{k+1} = G(x^k), \label{eq:fixed-point}
\end{equation}
which we hope will converge to a fixed point, i.e. $x^* = G(x^*)$.
We often approach convergence analysis through the
{\em error iteration} relating the error $e^k = x^k-x^*$ at
successive steps:
\begin{equation}
  e^{k+1} = G(x^* + e^k)-G(x^*).
\end{equation}

\paragraph{Model-based methods}
Most nonquadratic problems are too hard to solve directly.  On the other
hand, we can {\em model} hard nonquadratic problems by simpler (possibly
linear) problems as a way of building iterative solvers.  The most
common tactic --- but not the only one! --- is to approximate the
nonlinear function by a linear or quadratic function and apply all the
things we know about linear algebra.  We will return to this idea in
when we discuss Newton-type methods for optimization.

\section{Gradient descent}

One very simple iteration is {\em steepest descent}
or {\em gradient descent}:
\begin{equation} \label{iter:gd}
  x^{k+1} = x^k - \alpha_k \nabla \phi(x^k)
\end{equation}
where $\alpha_k$ is the {\em step size}, chosen adaptively or
with some fixed schedule.

To understand the convergence of this method, consider gradient
descent with a fixed step size $\alpha$ for the quadratic model problem
\[
  \phi(x) = \frac{1}{2} x^T A x + b^T x + c
\]
where $A$ is symmetric positive definite.  
We have computed the gradient for a quadratic before:
\[
  \nabla \phi(x) = Ax + b,
\]
which gives us the iteration equation
\[
  x_{k+1} = x_k - \alpha (A x_k + b).
\]
Subtracting the fixed point equation
\[
  x_* = x_* - \alpha (A x_* + b)
\]
yields the error iteration
\[
  e_{k+1} = (I-\alpha A) e_k.
\]
If $\{ \lambda_j \}$ are the eigenvalues of $A$, then the
eigenvalues of $I-\alpha A$ are $\{ 1-\alpha \lambda_j \}$.
The spectral radius of the iteration matrix is thus
\[
  \max \{ |1-\alpha \lambda_j| \}_j =
  \max \left( |1-\alpha \lambda_{\min}|, |1-\alpha \lambda_{\max}| \right).
\]
The iteration converges provided $\alpha < 2/\lambda_{\max}$, and the
optimal $\alpha$ is
\[
  \alpha_* = \frac{2}{\lambda_{\min} + \lambda_{\max}},
\]
which leads to the spectral radius
\[
  1 - \frac{2 \lambda_{\min}}{\lambda_{\min} + \lambda_{\max}} =
  1 - \frac{2}{1 + \kappa(A)}
\]
where $\kappa(A) = \lambda_{\max}/\lambda_{\min}$ is the condition
number for the (symmetric positive definite) matrix $A$.  If $A$
is ill-conditioned, then, we are forced to take very small steps
to guarantee convergence, and convergence may be
heart breakingly slow.  We will get to the minimum in the long run
--- but, then again, in the long run we all die.

\section{The Benefits of Slow Convergence}

How steepest descent behaves on a quadratic model
is how it behaves generally: if $x_*$ is a
strong local minimizer of some general nonlinear $\phi$, then gradient
descent with a small enough step size will converge locally to
$x_*$.  But if $H_{\phi}(x_*)$ is ill-conditioned, then one has to
take small steps, and convergence can be quite slow.

Somewhat surprisingly, sometimes we {\em want} this slow convergence.
To illustrate why, consider the {\em Landweber iteration}, which
is steepest descent iteration applied to linear least squares
problems:
\[
  x^{k+1} = x^k - \alpha_k A^T (Ax^k-b).
\] 
If we start from the initial guess $x^0 = 0$ and let the step size
be a fixed value $\alpha_k = \alpha$, we have the subsequent steps
\begin{align*}
  x^1 &= \alpha A^T b \\
  x^2 &= (I-\alpha A^T A) \alpha A^T b + \alpha A^T b \\
  x^3 &= (I-\alpha A^T A)^2 \alpha A^T b +
         (I-\alpha A^T A) \alpha A^T b +
         \alpha A^T b
\end{align*}
and so forth.
That is, each step is a partial sum of a {\em Neumann series},
which is the matrix generalization of the geometric series
\[
  \sum_{j=0}^k z^j = (1-z^{k+1})(1-z)^{-1} \rightarrow (1-z)^{-1}
  \mbox{ as } k \rightarrow \infty \mbox{ for } |z| < 1.
\]
Using the more concise expression for the partial sums of the
Neumann series expansion, we have
\begin{align*}
  x^{k+1}
  &= \sum_{j=0}^k (I-\alpha A^TA)^j \alpha A^T b \\
  &= (I-(I-\alpha A^TA)^{k+1}) (\alpha A^TA)^{-1} \alpha A^T b \\
  &= (I-(I-\alpha A^TA)^{k+1}) A^\dagger b.
\end{align*}
Alternately, we can write the iterates in terms of the singular value
decomposition with a filter for regularization:
\[
x^{k+1} = V \tilde{\Sigma}^{-1} U^T b, \quad
\tilde{\sigma}_j^{-1} = (1-(1-\alpha \sigma_j^2)^{k+1})) \sigma_j^{-1}.
\]
Hence, rather than running the Landweber iteration to convergence,
we typically stop when $k$ is large enough so that the filter is
nearly the identity for large singular values, but is small enough
so that the influence of the small singular values is suppressed.

\section{Preconditioning stationary iterations}

While the slow convergence of iterations like Landweber has some
surprising advantages, sometimes it is just a pain.  However, we can
speed up these transformations by {\em preconditioning} the problem.
That is, rather than applying the Landweber iteration to the problem
\[
  \min_{x} \|Ax-b\|^2
\]
we instead consider
\[
  \min_{x = R^{-1} y} \|A \tilde{R}^{-1} y - b\|^2
\]
where $\tilde{R}^{-1}$ is easy to apply (e.g. because $\tilde{R}$
might be chosen to be upper triangular) and $A \tilde{R}^{-1}$ has a
much smaller condition number than $A$.  In the extreme case where
$\tilde{R}$ is the $R$ factor in a QR factorization of $A$, we would
be able to solve the resulting problem by one step of Landweber with
step length one.  But we can often do pretty well even when far from
the case where $A \tilde{R}^{-1}$ has orthonormal columns.  This type
of re-scaling of the problem to encourage fast convergence is often
called {\em preconditioning}.

\section{Krylov subspace iterations}

% LSQR and LSMR + preconditioning (will talk about this with RandNLA)

We now consider a more general class of iterative methods that
{\em accelerates} the convergence of methods like Landweber.
There are many ways to derive these accelerated methods (Krylov
subspace methods).  We deliberately choose a somewhat unorthodox
description that highlights the connections to other accelerated
solvers we will encounter later in the class, as well as to our final
unit on learning dynamical systems from data.

Let's momentarily consider the case of solving a linear system $Ax = b$,
keeping the special case of the normal equations in the back of our
minds.  A {\em stationary iteration} has the form
\[
  M x_{k+1} = K x_k + b
\]
where $A = M - K$ is sometimes called a {\em splitting}.  The typical
way that we analyze such iterations is to subtract the fixed point
equation from the iteration, yielding
\[
  M (x_{k+1}-x_*) = K (x_{k} - x_*)
\]
or $e_{k+1} = (M^{-1} K) e_k = (M^{-1} K)^k e_0$ where $e_k = x_k-x_*$
is the error at step $k$.  The Landweber iteration with fixed step
size is an example of a stationary iteration.

Stationary iterations are an example of a linear time-invariant (LTI)
dynamical system in discrete time.  The dynamics can be described
entirely by the eigenvalue decomposition of the iteration matrix
$R = M^{-1} K$.  Even when the error is guaranteed to decay, general it
may decay quickly in some directions (associated with eigenvalues of
small magnitude) and slowly in others (associated with eigenvalues
with magnitude near 1).  We can get rid of the slowly-decaying
directions (also called modes) of the error by {\em filtering} them
from the sequence.  Unfortunately, the simplest way to construct such
a filter in advance involves knowing where the eigenvalues of the
iteration matrix lie, at least approximately, and that's often tricky.

An alternative approach is to ``learn'' the filter from the data by
considering all possible filtered sequences, i.e. we consider
\[
  \tilde{x}_k = \sum_{j=0}^k \beta_{jk} x_k
\]
for some to-be-determined set of coefficients $\beta$.  Simplifying
slightly by taking $x_0 = 0$, we would have that $\tilde{x}_k$ lies in
the $k+1$-dimensional {\em Krylov subspace}
\[
  \mathcal{K}_{k+1}(R, b)
  = \mathrm{sp}\{ b, R b, R^2 b, \ldots, R^k b \}
  = \mathrm\{ p(R) b : p \in \mathcal{P}_k \}
\]
where $\mathcal{P}_k$ is the space of polynomials of degree at most
$k$.

It turns out that Krylov subspaces often contain very good
approximations to the solution to a linear system.  Different Krylov
subspace methods choose the ``best'' approximate solution to $Ax = b$
in a Krylov subspace using different criteria.  When $A$ is symmetric
and positive definite, we might minimize a quadratic
form $\phi(z) = z^T Az/2 - z^T b$ over the subspace; this gives us
the {\em method of conjugate gradients} (CG).  Or we might minimize
the residual $\|Az-b\|$ over all $z$ in the subspace; this gives us
the minimum residual method (MINRES) in the symmetric case, or the
generalized minimal residual method (GMRES) in the nonsymmetric case.

Applying CG and MINRES to the normal equations gives an algorithms
that, while effective in principle, are not as numerically stable as
one might like.  One can rearrange these algorithms to get more stable
versions specifically for least squares problems; CG in this setting
is the basis for LSQR, and MINRES is the basis for LMRES.

\section{Gradient descent with errors}

Before we turn to stochastic gradient descent, let us instead look at
how to analyze {\em gradient descent with errors}.  In particular,
consider the iteration
\[
  x^{k+1} = x^k - \alpha_k p^k
\]
where
\[
  p^k = \nabla \phi(x^k) + u^k
\]
for some error $u^k$ that is ``small.''  As before, let's keep things
simple by looking how this iteration behaves for a quadratic model
problem with a fixed step size, i.e.
\[
  x^{k+1} = x^k - \alpha (Ax^k + b + u^k).
\]
Subtracting $x^*$ from both sides gives the error iteration
\[
  e^{k+1} = (I-\alpha A) e^k - \alpha u^k.
\]  
A little mumbling over the iteration gives us
\[
  e^{k+1} = (I-\alpha A)^{k+1} e^0 - \alpha \sum_{j=0}^k (I-\alpha A)^{k-j} u^j.
\]

In order to analyze the second term in this iteration, we need some
additional sort of control.  In the simplest case, that control might
be deterministic.  For example, if we can guarantee that
$\|u^k\| \leq C \gamma^{-k}$, then we have the bound
\[
  \left\| \sum_{j=0}^k  (I-\alpha A)^{k-j} u_j \right\| \leq
  C \gamma^{-k} \sum_{j=0}^k \left( \gamma \|(I-\alpha A)\|\right)^{k-j} \leq
  \frac{C \gamma^{-k-1}}{1-\gamma \|I-\alpha A\|}.
\]
Hence, we can make the iteration converge with inaccurate gradients,
as long as the accuracy improves sufficiently quickly with time.

We will pick up this iteration again next time under the assumption
that the errors are random, which is what happens in the
{\em stochastic gradient} method.

\end{document}
